\documentclass{article}

\usepackage[utf8]{inputenc}
\usepackage{titlesec}
\usepackage{titling}
\usepackage[margin=1in]{geometry}
\usepackage[american]{babel}
\usepackage{csquotes, xpatch}
\usepackage{hyperref}
\usepackage{xcolor}
% Define base style for publist (before loading biblatex)
\newcommand*\publistbasestyle{custom-authoryear}
\usepackage[bibstyle=publist, backend=biber, plauthorhandling=highlight, defernumbers=true, linktitleall=true, doi=false, isbn=false, url=false, boldyear=true, marginyear=false, plnumbered=false, sorting=ydnt, natbib=true, minbibnames=3, maxbibnames=3]{biblatex}

% Bibiography formatting
\setlength{\bibitemsep}{.75\baselineskip plus 0.5\baselineskip}
% \setlength{\bibhang}{.5em}
% No letters after years
\AtEveryCitekey{\clearfield{extradate}}
\AtEveryBibitem{\clearfield{extradate}}

\addbibresource{pubs.bib}
\addbibresource{talks.bib}
\addbibresource{posters.bib}

% Define formatting for publication margin year
\renewcommand*\plmarginyear[1]{
  \flushright{#1}
}

% Specify author name to bold in publication list
\plauthorname[Owen][]{Morgan}
\plauthorname[Owen P.][]{Morgan}
\plauthorname[O][]{Morgan}
\plauthorname[O. P.][]{Morgan}

% Define function for changing margin size (for publist)
\def\changemargin#1#2{\list{}{\rightmargin#2\leftmargin#1}\item[]}
\let\endchangemargin=\endlist 

% Define the default entry command - year, content
\newcommand{\entry}[5]{

  \begin{minipage}[t]{.15\textwidth}
    \begin{flushright}
      \hfill {#1}
    \end{flushright}
  \end{minipage}
  \hfill\vline\hfill
  \begin{minipage}[t]{.80\textwidth}
    \textbf{#2}

    \ifx&#3&
      {#3}\textit{#4}
    \else
      {#3, }\textit{#4}

    \fi
    \footnotesize{#5}
  \end{minipage}\\\vspace{.25cm}
}

% Define award entry command - year, title, details
\newcommand{\awardentry}[3]{
  \begin{minipage}[t]{.15\textwidth}
    \begin{flushright}
      \hfill {#1}
    \end{flushright}
  \end{minipage}
  \hfill\vline\hfill
  \begin{minipage}[t]{.80\textwidth}
    \textbf{#2}

    \footnotesize{#3}
  \end{minipage}\\\vspace{.25cm}
}

% Define the skill entry command - title, details
\newcommand{\skillentry}[2]{
  {\hspace{2em}\textbf{#1}:}
  {#2}
  \vspace{.25cm}
}

% No indentation
\setlength{\parindent}{0in}

% Macros
\newcommand{\cu}{Cornell University}
\newcommand{\sjc}{St. John's College}
\newcommand{\jhusom}{Johns Hopkins University School of Medicine}

% Hyperlinks
\let\oldhref\href
\renewcommand{\href}[3][blue]{\oldhref{#2}{\color{#1}{#3}}}


% Section formatting
\titleformat{\section}
{\large\bfseries}
{}
{0em}
{}[\titlerule]

\titleformat{\subsection}
{\center\large\bfseries}
{}
{0em}
{}

\titleformat{\subsubsection}[runin]
{\bfseries}
{\hspace{0em}}
{0em}
{}[:]

\titlespacing{\subsubsection}
{0em}{.25em}{.25em}

\renewcommand{\maketitle}{
  \begin{minipage}[t]{.5\textwidth}
    \begin{flushleft}
      \textit{Last updated \thedate}
    \end{flushleft}
  \end{minipage}
  \begin{minipage}[t]{.5\textwidth}
    \begin{flushright}
      {\huge\bfseries
      \theauthor}

      \vspace{.25em}

      \cu, Department of Human Development

      Experience and Cognition Lab

      opm6@cornell.edu

    \end{flushright}
  \end{minipage}
}

\author{Owen Morgan}
\title{CV}

% Update date set to last compile:
\date{\today}

\begin{document}

\maketitle

\section{Education}

\entry
{2020 -- Now}
{PhD Student}
{\cu}{Ithaca, New York}
{Advisor: Daniel Casasanto. Major in Developmental Psychology, minor in Cognitive Science.}

\entry
{2013 -- 2017}
{BA in Liberal Arts}
{\sjc}{Annapolis, Maryland}
{All-required curriculum based on closely reading original sources; equivalent to a double major in Philosophy, and the History of Mathematics and Science.}

\section{Professional Experience}

\entry
{2017 -- 2020}
{Research Assistant, Cognitive Neuropsychiatric Research Laboratory}
{\jhusom}{Baltimore, MD}
{Studied the cerebellum’s role in cognition and implemented several neuroimaging projects under the mentorship of Cherie Marvel, PhD.}

\entry
{Summers 2016 2015}
{Research Assistant, Blumenfeld Lab}
{Yale Medical School}{New Haven, CT}
{Studied conscious visual perception under the mentorship of Hal Blumenfeld, MD, PhD.}

\entry
{2014 -- 2017}
{Lab Assistant}
{\sjc}{Annapolis, MD}
{Worked with professor to lead class discussions; set up, demonstrated, and explained historical experiments from biology, physics, and chemistry.}

\section{Publications}


% Change margins for publist's marginal years
% \begin{changemargin}{0in}{0in} 

\newrefsection[pubs]
\nocite{*}
\printbibliography[heading=none]

% Restore margins
% \end{changemargin}



\section{Presentations}

\subsection{Talks}
\newrefsection[talks]
\nocite{*}
\printbibliography[heading=none]

\subsection{Posters}
\newrefsection[talks]
\nocite{*}
\printbibliography[heading=none]


\section{Honors and Awards}

\awardentry{2020}{Travel Award for 9th Ataxia Investigators Meeting (postponed to May 2021, virtual)}
{Awarded to abstracts selected for “hot chair” presentation.}

\awardentry{Summer 2017}{Pathways Fellowship}
{Awarded to qualified students from St. John’s College for special or prerequisite courses for graduate study or careers. Funded Neurobiology course at Harvard Extension Summer School.}

\awardentry{2016 2015}{Hodson Grant}
{Awarded to qualified students from St. John’s College for research and internships. Funded summer research on conscious visual perception at Yale University.}

\section{Teaching and Mentorship}

\entry{Spring 2021}{TA, Adolescence (HD 1170)}
{\cu}{Ithaca, New York}
{Instructor: Robert Sternberg.}

\entry{Fall 2020}{TA, Human Brain and Mind: Introduction to Cognitive Neuroscience\linebreak(HD 2200)}
{\cu}{Ithaca, New York}
{Instructor: Daniel Casasanto.}

\entry{2018 -- 2020}{Research Assistant Mentorship}
{\jhusom}{Baltimore, Maryland}
{Supervised and mentored Bronte Wen, Deeya Bhattacharya, Nikita Gupta on fMRI analysis projects.}

% \section{Leadership and Service}
%
% \entry{2017 -- 2019}{Leader, Study Group on Existential Phenomenology and Cognitive Science}
% {}{Baltimore, Maryland}
% {\\ Discussion group with St. John’s College alumni and colleagues from Johns Hopkins. Readings included Maurice Merleau-Ponty’s The Phenomenology of Perception and Shaun Gallagher’s Enactivist Interventions.}
%
% \entry{2015 -- 2017}{President, Waltz Committee}
% {\sjc}{Annapolis, Maryland}
% {Organized dances and lessons for students, alumni, and community; managed budget.}
%
% \entry{2014 -- 2017}{President, Farming and Gardening Club}
% {\sjc}{Annapolis, Maryland}
% {Started campus vegetable farm; organized and hosted events; managed budget.}
%
% \entry{2014 -- 2015}{Representative, Delegate Council}
% {\sjc}{Annapolis, Maryland}
% {Student branch of school government. Wrote and voted on legislation; managed budget for student activities and events.}

\section{Technical Skills}

\skillentry{fMRI}
{Collection (3T, 7T), preprocessing, whole brain and ROI analyses (SPM)}

\skillentry{Eye-tracking}{Collection, preprocessing, analysis}

\skillentry{Scripting}{R, Python, MATLAB}

\skillentry{Task development}
{Python/PsychoPy, ePrime}

\skillentry{Data visualization}{R/ggplot, Inkscape/Illustrator}

\skillentry{Markup}{\LaTeX, RMarkdown}

\skillentry{OS}{Linux, Windows, MacOS}

\section{Spoken Languages}
\skillentry{English}{Native}

\skillentry{Spanish}{Full professional proficiency (ILR scale)}

\end{document}

